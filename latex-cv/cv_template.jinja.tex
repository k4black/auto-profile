%%%%%%%%%%%%%%%%%
% This is an sample CV template created using altacv.cls
% (v1.6.5, 3 Nov 2022) written by LianTze Lim (liantze@gmail.com). Compiles with pdfLaTeX, XeLaTeX and LuaLaTeX.
%
%% It may be distributed and/or modified under the
%% conditions of the LaTeX Project Public License, either version 1.3
%% of this license or (at your option) any later version.
%% The latest version of this license is in
%%    http://www.latex-project.org/lppl.txt
%% and version 1.3 or later is part of all distributions of LaTeX
%% version 2003/12/01 or later.
%%%%%%%%%%%%%%%%

%% Use the "normalphoto" option if you want a normal photo instead of cropped to a circle
% \documentclass[10pt,a4paper,normalphoto]{altacv}

%\documentclass[10pt,a4paper,ragged2e,withhyper,debug]{altacv}
\documentclass[10pt,a4paper,ragged2e,withhyper]{altacv}
%% AltaCV uses the fontawesome5 and packages.
%% See http://texdoc.net/pkg/fontawesome5 for full list of symbols.

% Change the page layout if you need to
\geometry{left=1.25cm,right=1.25cm,top=1.25cm,bottom=1cm,columnsep=1.0cm}
%\geometry{left=1cm,right=1cm,marginparsep=1.2cm,top=1.25cm,bottom=1.25cm,columnsep=1.2cm}


\setlength{\footskip}{1.2em}

%% new: for using \today
\RequirePackage[en-US]{datetime2}



% The paracol package lets you typeset columns of text in parallel
\usepackage{paracol}

% Change the font if you want to, depending on whether
% you're using pdflatex or xelatex/lualatex
\ifxetexorluatex
  % If using xelatex or lualatex:
  \setmainfont{Roboto Slab}
  \setsansfont{Lato}
  \renewcommand{\familydefault}{\sfdefault}
\else
  % If using pdflatex:
  \usepackage[rm]{roboto}
  \usepackage[defaultsans]{lato}
  % \usepackage{sourcesanspro}
  \renewcommand{\familydefault}{\sfdefault}
\fi

% Change the colours if you want to
\definecolor{Accent}{HTML}{2B7A89}
\definecolor{LightAccent}{HTML}{2da0c5}
\definecolor{SlateGrey}{HTML}{2E2E2E}
\definecolor{Grey}{HTML}{505050}
\definecolor{LightGrey}{HTML}{AAAAAA}
\colorlet{name}{black}
\colorlet{tagline}{SlateGrey}
\colorlet{heading}{Accent}
\colorlet{headingrule}{Accent}
\colorlet{subheading}{LightAccent}
\colorlet{accent}{LightAccent}
\colorlet{emphasis}{SlateGrey}
\colorlet{body}{Grey}
\colorlet{light}{LightGrey}

% Change some fonts, if necessary
\renewcommand{\namefont}{\Huge\bfseries}
\renewcommand{\personalinfofont}{\footnotesize}
\renewcommand{\cvsectionfont}{\LARGE\bfseries}
\renewcommand{\cvsubsectionfont}{\large\bfseries}


% Change the bullets for itemize and rating marker
% for \cvskill if you want to
\renewcommand{\itemmarker}{{\small\textbullet}}
\renewcommand{\ratingmarker}{\faCircle}

%% Use (and optionally edit if necessary) this .tex if you
%% want to use an author-year reference style like APA(6)
%% for your publication list
% \input{pubs-authoryear.cfg}

%% Use (and optionally edit if necessary) this .tex if you
%% want an originally numerical reference style like IEEE
%% for your publication list
\input{pubs-num.cfg}

%% sample.bib contains your publications
\addbibresource{sample.bib}


\usepackage[none]{hyphenat}


\begin{document}
\pagestyle{fancy}


  \name{(( bio.name ))}
%  \tagline{Your Position or Tagline Here}
  %% You can add multiple photos on the left or right
  \photoR{2.8cm}{photo_spring_2023.jpeg}
  % \photoL{2.5cm}{Yacht_High,Suitcase_High}

  \personalinfo{%
    % Not all of these are required!
    \email{(( bio.email ))}
  %  \phone{000-00-0000}
  %  \mailaddress{Åddrésş, Street, 00000 Cóuntry}
    \location{(( bio.location ))}
    \homepage{(( bio.website ))}
  %  \twitter{@twitterhandle}
    \linkedin{(( bio.linkedin ))}
    \github{(( bio.github ))}
  %  \orcid{0000-0000-0000-0000}
    %% You can add your own arbitrary detail with
    %% \printinfo{symbol}{detail}[optional hyperlink prefix]
    % \printinfo{\faPaw}{Hey ho!}[https://example.com/]
    %% Or you can declare your own field with
    %% \NewInfoFiled{fieldname}{symbol}[optional hyperlink prefix] and use it:
    % \NewInfoField{gitlab}{\faGitlab}[https://gitlab.com/]
    % \gitlab{your_id}
    %%
    %% For services and platforms like Mastodon where there isn't a
    %% straightforward relation between the user ID/nickname and the hyperlink,
    %% you can use \printinfo directly e.g.
    % \printinfo{\faMastodon}{@username@instace}[https://instance.url/@username]
    %% But if you absolutely want to create new dedicated info fields for
    %% such platforms, then use \NewInfoField* with a star:
    % \NewInfoField*{mastodon}{\faMastodon}
    %% then you can use \mastodon, with TWO arguments where the 2nd argument is
    %% the full hyperlink.
    % \mastodon{@username@instance}{https://instance.url/@username}
  }

  \makecvheader{}
  %\begin{fullwidth}
  %    \makecvheader{}
  %\end{fullwidth}
  \makecvfooter
      {Autogenerated with \ihref[\faGithub]{((project_url))}{((project_name))}}
      {}
      {Updated \today}%

  %% Depending on your tastes, you may want to make fonts of itemize environments slightly smaller
  % \AtBeginEnvironment{itemize}{\small}

  %% Set the left/right column width ratio to 6:4.
  \columnratio{0.62}

  % Start a 2-column paracol. Both the left and right columns will automatically
  % break across pages if things get too long.
  \begin{paracol}{2}

    %%%%%%%%%%%%%%%%%%%%%%%%% LEFT COLUMN %%%%%%%%%%%%%%%%%%%%%%%%%

    \cvsection{Experience}
    (# for experience in experience_list #)
        \cvevent
            { (( experience.role )) }
            { \ihref{ (( experience.url )) }{ (( _process_text_to_latex(experience.company) )) } }
            {(( experience.start )) --\,(( experience.end ))}
            {(( experience.location ))}%
        \cvbox[\linewidth]{%
            \begin{itemize}
                (#- for description in experience.description #)
                  \item (( _process_text_to_latex(description) ))
                (#- endfor #)
            \end{itemize}%
            \hspace{1em}\cvsmalltagline{ (#- for tag in experience.tags -#) (( tag ))((", " if not loop.last else "")) (#- endfor -#) }\smallskip
        }\hfill\par%%
        \bigskip%
    (# endfor #)

    \vspace{-0.3em} % TODO: fix
    \cvsection{Education}
    (# for education in education_list #)
        \cvevent
            { (( education.degree )) }
            { \ihref{ (( education.url )) }{ (( _process_text_to_latex(education.institution) )) } }
            {(( education.start )) --\,(( education.end ))}
            {(( education.location ))}%
        \cvbox[\linewidth]{(( _process_text_to_latex(education.description) ))}\hfill\par%%
        \bigskip%
    (# endfor #)

    \smallskip\divider%
    \vspace{-0.6em} % TODO: fix

    \cvevent
        {Courses and Certificates}
        {}
        {}
        {}%
    \begin{itemize}
        \setlength\itemsep{0em}
        (#- for certificate in certificates_list #)
            \item (( certificate ))
        (#- endfor #)
    \end{itemize}


%    % use ONLY \newpage if you want to force a page break for
%    % ONLY the current column
%    \newpage

%    \cvsection{Publications}
%
%    %% Specify your last name(s) and first name(s) as given in the .bib to automatically bold your own name in the publications list.
%    %% One caveat: You need to write \bibnamedelima where there's a space in your name for this to work properly; or write \bibnamedelimi if you use initials in the .bib
%    %% You can specify multiple names, especially if you have changed your name or if you need to highlight multiple authors.
%    \mynames{Lim/Lian\bibnamedelima Tze,
%      Wong/Lian\bibnamedelima Tze,
%      Lim/Tracy,
%      Lim/L.\bibnamedelimi T.}
%    %% MAKE SURE THERE IS NO SPACE AFTER THE FINAL NAME IN YOUR \mynames LIST
%
%    \nocite{*}
%
%    \printbibliography[heading=pubtype,title={\printinfo{\faBook}{Books}},type=book]
%    \printbibliography[heading=pubtype,title={\printinfo{\faFile*[regular]}{Journal Articles}},type=article]
%    \printbibliography[heading=pubtype,title={\printinfo{\faUsers}{Conference Proceedings}},type=inproceedings]

    %%%%%%%%%%%%%%%%%%%%%%%%% RIGHT COLUMN %%%%%%%%%%%%%%%%%%%%%%%%%
    %% Switch to the right column. This will now automatically move to the second
    %% page if the content is too long.
    \switchcolumn

    \vspace{-0.2em} % TODO: fix
    \cvsection{Summary}
    \vspace{0.6em} % TODO: fix
    \textbf{(( short_summary ))}\par%
    \smallskip%
    (( summary ))
    \medskip

    \cvsection{Skills}
    \cvbox{
        \begin{itemize}
            \setlength\itemsep{0.6em}
            (#- for skill_group in skills_list #)
                \item \textcolor{emphasis}{(( skill_group.group ))}\par
                (#- for skill in skill_group.tags -#)
                    \cvsmalltag[(( skill.level ))0]{(( skill.name ))}
                (#- endfor -#)
            (#- endfor #)
        \end{itemize}
    }%
    \medskip


    \cvsection{Publications}
    \cvbox{
        \begin{itemize}
            (#- for publication in publication_list #)
                \item \textit{\ihref{(( publication.url ))}{(( publication.title ))}} \,[(( publication.venue ))];
            (#- endfor #)
        \end{itemize}
    }%
    \medskip


    \cvsection{Achievements}
    \begin{itemize}
        (#- for achievement in achievements_list #)
            \item (( achievement ))
        (#- endfor #)
    \end{itemize}
    \medskip

    \cvsection{Personal}
    ((personal_summary))\\
     (#- for tag in personal_tags_list #) \cvsmalltag{((tag))} (#- endfor #)


  \end{paracol}


\end{document}
